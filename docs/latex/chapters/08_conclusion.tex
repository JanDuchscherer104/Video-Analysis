% 8. Conclusion and Discussion
\section{Conclusion and Discussion}
\label{sec:conclusion}
This seminar work provided a comprehensive theoretical analysis of trajectory prediction methods in autonomous driving. The investigation examined multiple approaches, from the raster-based CASPNet to the transformer-based CASPFormer and LMFormer architectures, identifying key design principles and implementation challenges. In our \section{sec:background}, we provided a detailed qualitative comparison of commonly used scene representation paradigms, paying particular attention to the differences between agent-centric and query-centric approachs, examining their respective strengths and weaknesses. This analysis highlighted how query-centric methods enable permutation and $\mathrm{SE}(2)$-equivariance while supporting streaming inference and joint prediction, contrasting with the limitations of agent-centric frameworks bound to privileged reference frames.

The practical work achieved significant outcomes through successful refactoring and integration of the UniTraj framework with modern deep learning infrastructure. We established a robust experimental pipeline using PyTorch Lightning and achieved a brier-minFDE of 1.98 compared to the MTR benchmark of 2.08. However, our assessment revealed critical issues in the original framework including poor coding standards, inadequate documentation, and suboptimal utilization of modern ML frameworks. Despite these obstacles, our mitigation strategies established a reproducible training pipeline that demonstrates the effectiveness of query-centric representations for trajectory forecasting.


\subsection{Reflection on Initial Objectives}
\label{sec:conclusion_objectives}

The initial aim of this seminar work was to implement a minimal, simplified model for multi-agent joint trajectory prediction in autonomous driving. However, the complexity of existing algorithms and computational constraints necessitated a significant scope reduction to focus on a comparative analysis between different algorithms instead of implementing a "Very-Smol-Single-Agent-Motion-Forecasting" algorithm.

The complexity of this analysis presented throughout this report, particularly the comparative study in Chapter~\ref{sec:background} examining agent-centric versus query-centric paradigms and the detailed description of the different models in Chapter~\ref{ch:model_architecture}, demonstrates the depth of investigation achieved. While the original implementation goal proved unattainable, the work achieved a thorough understanding of these fundamental approaches.

Engaging with state-of-the-art transformer architectures deepened our understanding of attention mechanisms for encoding spatial and temporal relationships. The work explored how cross attention fuses different modalities and examined design choices such as mode queries and anchors. Applying a Geometric Deep Learning perspective to the query-centric paradigm fostered understanding of concepts including isomorphisms and fiber bundles. The experimental work demonstrated practical application of these concepts, despite scope limitations to single-agent prediction rather than the originally planned multi-agent joint forecasting.

This seminar work succeeded in building conceptual foundations and developing a critical perspective on current research directions. The experience highlighted the importance of clear abstractions and robust data handling practices while exposing the inherent challenges of adapting complex research frameworks during this seminar.

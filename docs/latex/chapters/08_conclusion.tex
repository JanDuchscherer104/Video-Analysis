% 8. Conclusion and Discussion
\section{Conclusion and Discussion}
\label{sec:conclusion}
This seminar work provided a comprehensive theoretical analysis of selected trajectory prediction methods in autonomous driving. The investigation examined multiple approaches, from the raster-based CASPNet to the transformer-based CASPFormer, MTR and LMFormer architectures, elucidating key design principles and interesting architectural features. In our \autoref{sec:background}, we provided a detailed qualitative comparison of commonly used scene representation paradigms, paying particular attention to the differences between agent-centric and query-centric approachs, examining their respective strengths and weaknesses. This analysis highlighted how query-centric methods enable permutation and $\mathrm{SE}(2)$-invariance while supporting streaming inference and joint prediction, contrasting with the limitations of agent-centric frameworks bound to privileged reference frames.

Our practical work resulted in significant improvements to the UniTraj framework, refactoring it in accordance with the principles of modern deep learning and fixing various issues.
We established a robust experimental pipeline using PyTorch Lightning and achieved a brier-minFDE of 1.98 compared to the MTR benchmark of 2.08.


\subsection{Reflection on Initial Objectives}
\label{sec:conclusion_objectives}

The initial aim of this seminar work was to implement a minimal, simplified model for \emph{joint multi-agent} trajectory prediction. However, the complexities of the chosen task became apparent incrementally throughout the project.
The inherent intricacies of the existing algorithms, and working with a framework like UniTraj, as well as limited computational constraints necessitated a significant scope reduction to focus on a comparative analysis between different algorithms instead of implementing a ``Very-Smol-Single-Agent-Motion-Forecasting'' algorithm.

The complexities within the field of multi-modal trajectory prediction are presented throughout this report, particularly the comparative study in Chapter~\ref{sec:background} examining agent-centric versus query-centric paradigms and the detailed description of the different models in Chapter~\ref{ch:model_architecture}, demonstrates the depth of the field, as well as the comprehensiveness of our investigation. While the original implementation goal proved unattainable, the work achieved a thorough understanding of various important concepts and design choices, whose implications reach far beyond the domain of motion forecasting.

Engaging with state-of-the-art transformer architectures deepened our understanding of attention mechanisms for encoding spatial and temporal relationships. The work explored how cross attention fuses different modalities and examined design choices such as mode queries and anchors. Applying a Geometric Deep Learning perspective to the query-centric paradigm fostered understanding of concepts including isomorphisms and fiber bundles. The experimental work demonstrated practical application of these concepts.

This seminar work succeeded in building conceptual foundations and developing a critical perspective on current research directions. The experience highlighted the importance of clear abstractions and robust data handling practices while exposing the inherent challenges of adapting complex research frameworks during this seminar.

Ultimately, our ambitious choice of problem revealed itself as disproportionately complex for a single-semester module, creating considerable pressure in the closing weeks. Acknowledging this mis-scoping and persevering through the stress it caused has become a formative lesson in realistic project framing that complements the countless theoretical insights we gained.

\newpage
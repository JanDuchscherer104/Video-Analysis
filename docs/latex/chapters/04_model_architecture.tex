

% 4. Model Architecture and Functional Decomposition
\chapter{Model Architecture and Functional Decomposition}
\label{ch:model_architecture}

\section{MTR Architectural Structrue}
\label{sec:model_mtr_architecture}
A formal description of the MTR's modular, Transformer-based architecture. (Inlcude architectural diagram)

\section{Input Encoding}
\label{sec:model_input_encoding}
    \subsection{Polyline Encoders}
    Processing of map polyline features.
	\subsection{Agent Encoders}
    Processing of agent historical states.

\section{Trajectory Decoding}
\label{sec:model_decoding}
    \subsection{Transformer Encoder Module}
    Aggregation and contextualization of input feature representations.
    \subsection{Dynamic Map Collection Strategy}
    Mechanism for adaptive incorporation of relevant map elements.
    \subsection{Motion Decoder Module (Transformer Decoder Layer)}
    \begin{itemize}
        \item \textbf{Intention-Driven Querying:} Role of Motion Query Pairs ($[Q_I, Q_S^j]$) in guiding trajectory generation.
        \item \textbf{Goal-Conditioned Prediction:} Generation of trajectories conditioned on latent goals or intentions.
    \end{itemize}
    \subsection{Probabilistic Multimodal Output Generation}
    Use of Gaussian Mixture Models (GMMs) to represent a distribution over future trajectories.
    \subsection{Iterative Query Refinement}
    Elaborate on these MTR components based on literature.

\section{MTR Input-Output Formulation}
\label{sec:model_mtr_io}
    \begin{itemize}
        \item \textbf{Input Domain:} Historical spatiotemporal data of agents and associated environmental context (map features).
        \item \textbf{Output Domain:} A set of K ranked, plausible future trajectories, each associated with a probability score.
    \end{itemize}
\section{Data Methodology within UniTraj}
\label{sec:data_methodology}

This section presents the data processing methodology within the UniTraj framework~\cite{unitrajFeng2024}, covering the transformation from heterogeneous autonomous driving datasets to standardized tensor representations suitable for trajectory prediction models. UniTraj serves as a unified interface that harmonizes datasets from WOMD~\cite{WOMD2021}, Argoverse2~\cite{av2Wilson2023}, and nuScenes~\cite{caesar2020nuscenes} through the ScenarioNet format~\cite{scenarionetLi2023}.

\subsection{UniTraj\'s Data Processing Pipeline}
\label{ssec:data_pipeline}

The UniTraj preprocessing pipeline transforms raw, heterogeneous data into standardized, model-ready tensors through a multi-phase process. Each phase is implemented as a distinct algorithm, ensuring modularity and clarity. The following provides a qualitative overview of the pipeline; for detailed implementations, see the algorithms in \autoref{app:framework}.

\begin{description}
    \item[Phase 1: Temporal Window Extraction] Extracts fixed-length time windows from raw trajectories and applies frequency masking for uniform temporal sampling (\autoref{alg:phase1_temporal}).
    \item[Phase 2: Map Feature Processing] Converts raw map data (lanes, boundaries) into standardized polylines with uniform point density and semantic encodings (\autoref{alg:phase2_map}).
    \item[Phase 3: Agent Selection] Filters for relevant agents based on motion and observation quality to identify suitable prediction candidates (\autoref{alg:phase3_filtering}).
    \item[Phase 4: Coordinate Transformation] Transforms the entire scene into an agent-centric coordinate frame for each candidate, ensuring translation and rotation invariance (\autoref{alg:phase4_transform}).
    \item[Phase 5: Feature Assembly] Constructs comprehensive feature vectors for each agent, concatenating spatial, kinematic, and semantic attributes (\autoref{alg:phase5_features}).
    \item[Phase 6: Proximity Filtering \& Padding] Selects the \(N_{\max}\) closest agents and pads the agent tensor to a fixed size for batching (\autoref{alg:phase6_proximity}).
    \item[Phase 7: Map Tensorization] Segments, resamples, and selects the \(K_{\max}\) closest map polylines, creating a fixed-size map tensor (\autoref{alg:phase7_map_features}).
    \item[Phase 8: Future Processing] Processes and transforms the ground truth future trajectories for the center agent (\autoref{alg:phase8_future}).
    \item[Phase 9: DatasetItem Assembly] Assembles all processed tensors and masks into a final \texttt{DatasetItem}, the fundamental unit of the dataset (\autoref{alg:phase9_assembly}).
\end{description}

\subsection{The UniTraj DatasetItem and Batching}
\label{ssec:unitraj_dataset}

The output of the processing pipeline is a collection of \texttt{DatasetItem} instances, each encapsulating a complete, self-contained prediction scenario in a standardized format.

\paragraph{The \texttt{DatasetItem} Structure.}
A \texttt{DatasetItem} is a dictionary-like object that holds all the numpy arrays corresponding to a single, agent-centric view of a scene. This includes the historical agent states, the map geometry, ground truth future trajectories, and all associated validity masks. This atomic structure ensures that all information required for a single forward pass of a model is cleanly organized and easily accessible.

\paragraph{Batching with the \texttt{collate\_fn}.}
For model training and inference, individual \texttt{DatasetItem}s are grouped into batches by a PyTorch \texttt{DataLoader}. This process is orchestrated by a custom \texttt{collate\_fn}, which transforms a list of \texttt{DatasetItem}s into a single \texttt{BatchInputDict}. The function iterates through the keys of the items, stacks the corresponding numpy arrays along a new batch dimension, and converts them into PyTorch tensors. The resulting \texttt{BatchInputDict} is a dictionary of tensors, where each tensor represents a batch of data (e.g., a batch of agent histories, a batch of maps), ready for direct input into a prediction model.

The key tensors within this batch structure are described below. For complete tensor specifications, dimensions, and feature breakdowns, refer to Tables~\ref{tab:data_tensors},~\ref{tab:agent_types}, and~\ref{tab:polyline-types} in \autoref{app:notation}.

\paragraph{Dynamic Agent Representation.}
Agent trajectories are encoded in tensor \(\boldsymbol{X}_d \in \mathbb{R}^{B \times N_{\max} \times T_p \times F_{ap}}\), where \(B\) is the batch size. It contains comprehensive state information for up to \(N_{\max}\) agents over \(T_p\) historical timesteps. The agent feature dimension \(F_{ap}\) encompasses spatial coordinates, physical dimensions, one-hot encoded object types, temporal position embeddings, heading, and kinematic states. The corresponding validity mask \(\boldsymbol{M}_d \in \{0,1\}^{B \times N_{\max} \times T_p}\) indicates data availability.

\paragraph{Static Map Topology.}
High-definition map information is represented through tensor \(\boldsymbol{X}_s \in \mathbb{R}^{B \times K_{\max} \times L \times F_{map}}\), encoding up to \(K_{\max}\) polylines with \(L\) points each. The \(F_{map}\) features capture geometric and semantic properties, including polyline point coordinates, direction vectors, and lane type classifications. The validity mask \(\boldsymbol{M}_s \in \{0,1\}^{B \times K_{\max} \times L}\) handles variable map complexity.

\paragraph{Ground Truth and Auxiliary Data.}
Future trajectory targets for the center agent are provided in \(\boldsymbol{y}_c \in \mathbb{R}^{B \times T_f \times F_{af}}\), where \(F_{af}\) includes position and velocity over \(T_f\) prediction timesteps. The batch also contains metadata like difficulty scores and behavioral classifications for nuanced evaluation.

This unified, batched representation enables seamless and efficient model training across multiple datasets, supporting both single-dataset optimization and cross-dataset generalization studies as demonstrated in the UniTraj benchmark~\cite{unitrajFeng2024}.

\newpage